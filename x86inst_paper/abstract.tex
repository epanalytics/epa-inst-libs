\begin{it}

Binary instrumention toolkits enable insertion of additional code into an
executable in order to observe or modify the behavior of application runs. 
There are two main approaches to binary instrumentation: \textit{static} 
and \textit{dynamic} binary instrumentation. In this paper, we present X86ElfInstrumentor, 
a static binary instrumentation toolkit for Linux on x86/x86\_64 platforms. The X86ElfInstrumentor
is similar to the other toolkits in terms of how additional code is inserted. However, it uses wholesale
code relocation in order to remedy the difficulty created by the platforms' use
of variable-length instructions. Code relocation of this kind allows the
instrumentation tool to reorganize the application code in such a way that it
can use the fast but far-reaching constructs to transfer control
from the application to the instrumentation code rather than relying on multiple
jumps or interrupts for the transfer. Furthermore, the API provides a means of
allowing the tool developer to insert hand-coded assembly in a very lightweight
way rather than relying solely on the insertion of entire instrumentation functions.
Our static instrumentation toolkit enables implementation of efficient instrumentation tools, 
with overheads  for basic block counting that are an
average of 48\% of the overhead imposed by Pin, 18\% of the overhead imposed by
DynamoRIO, 10\% of the overhead of Valgrind, and 5\% of the overhead of Dyninst.
\end{it}
