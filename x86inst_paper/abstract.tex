\begin{it}

Binary instrumentation enables insertion of additional code into an
executable in order to observe or modify the behavior of application runs. 
There are two main approaches to binary instrumentation: static and dynamic
binary instrumentation. In this paper, we present PMaC's instrumentation toolkit (PIX), 
an efficient static  instrumentation toolkit for Linux on x86/x86\_64 platforms. PIX
is similar to the other toolkits in terms of how additional code is inserted. However, it uses whole function level
code relocation in order to remedy the difficulty created by the variable-length instruction set. Code relocation of this kind allows the
instrumentation tool to reorganize the application code in such a way that it
can use the fast  far-reaching constructs to transfer control
from the application to the instrumentation code rather than relying on multiple
jumps or interrupts for the transfer. Furthermore, the PIX API provides a means to
tool developers to insert lightweight hand-coded assembly
rather than relying solely on the insertion of entire instrumentation functions.
PIX also enables implementation of efficient instrumentation tools, 
with overheads for basic block counting that are an
average of 1.6x less than the overhead imposed by Pin, 4.7x less than the overhead imposed by
DynamoRIO, 7.8x less than the overhead imposed by Valgrind, and 75.6x less than the overhead imposed by Dyninst.

\end{it}
