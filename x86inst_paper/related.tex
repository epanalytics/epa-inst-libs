\section{Related Work}
\label{sec:Related}

ATOM \cite{srivastava1994atom} was one of the first and has remained one of the more popular static binary instrumentation toolkits
available. ATOM works in a way that is conceptually similar to PIX; instrumentation is performed on the compiled
binary prior to runtime, meaning that any overhead due to code analysis and code generation is incurred outside of
the instrumented application's run cycle. Unfortunately, ATOM is available only for the Alpha platform. Since this
processor is not being produced anymore, ATOM is no longer viable as a long-term solution for those who wish to
perform static, efficient instrumentation. 

Dyninst \cite{buck2000api} is a popular dynamic instrumentation toolkit that uses a technique called code patching
to perform instrumentation. Similar in concept to what is done in PIX, this technique replaces an
instruction from the application with a jump instruction to a function call stub and
instrumentation code. The key difference between PIX and Dyninst is that Dyninst performs all code patching
at runtime instead of prior to runtime. This has several advantages, including the ability to insert, remove
and customize instrumentation during runtime. But performing modification to the program at runtime also
may have a significant performance disadvantage, resulting in inefficient execution of the instrumented application.

Pin \cite{luk2005pin} is another popular dynamic binary instrumentation toolkit that uses a JIT-based (Just In Time compilation) approach to
instrumentation. This approach entails running the application on top of Pin, while Pin intercepts the
application at each natural control flow interruption in the program to perform instrumentation on the next part of the
program. For efficiency, Pin performs many optimizations including caching these instrumented sequences of code to allow for
re-use, chaining instrumented sequences of code together to avoid unnecessary tool intervention, and avoiding state protection
overheads whenever possible.

DynamoRIO \cite{bruening2004efficient} and Valgrind \cite{nethercote2007valgrind} are two other dynamic
binary instrumentation toolkits that use a JIT-based approach to instrumentation and operate in a similar fashion
to Pin. These toolkits offer certain functionality that is not available anywhere else. For example, Valgrind offers
support for a feature called shadow values \cite{nethercote2007shadow}, 
which can be used to create instrumentation tools that are difficult impossible to build without
their support. However, support of this kind entails a more heavyweight approach to instrumentation that is
unsatisfactory when efficiency is the primary goal.
