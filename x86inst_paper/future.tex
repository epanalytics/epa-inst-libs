\section{Future Work}

Despite the success in terms of efficiency, there are several additional techniques
that might make the instrumented code even more efficient in PIX. Because PIX relocates
the text to yeild extra space for the manipulation of the application functions, 
rather than inserting just a branch
that transfers control to the instrumentation code we have the opportunity to inline
the instrumentation code itsself
in order to reduce or eliminate the control interruptions  that otherwise must be taken 
when inserting the instrumentation code.

Currently PIX saves all general purpose registers around each function call and allows the
tool developer to state which registers are saved around instrumentation snippets. For even more efficient instrumentation
snippets, we could automatically detect which registers are killed by the instrumentation code and which are live at the entry point
of the instrumentation code, and automatically save only the ones that are alive. Similarly, we could perform register 
analysis in order to identify the instrumentation points where the machine state doesn't need to be saved around instrumentation functions. 

Finally, similar to Pin, we could perform liveness on the bits of the eflags/rflags register to determine whether the flag registers need to be saved and
restored at each instrumentation point. Saving and restoring state is a large portion of the overhead associated with performing
small tasks in the instrumentation code and we believe these optimizations will help reduce the overhead due to the instrumentation code even more.


