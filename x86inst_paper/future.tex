Despite some success in terms of efficiency, there are several more techniques
that might make the instrumented code even more efficient. Because we are relocating
the text to give ourselves as much space as possible, rather than inserting just a branch
that transfers control to the instrumentation code we have the opportunity to inline
the instrumentation code itsself in order to reduce  or eliminate the control interruptions
that otherwise must be taken when inserting instrumentation code.

We could also perform register liveness analysis in order to determine whether there is
state that doesn't need to be saved around instrumentation code or to guide the selection
of usable registers by the instrumentation tool developer. And similar to Pin, we could perform
liveness on the bits of the eflags/rflags register to determine whether it must be saved and
restored. Saving and restoring state is a large portion of the overhead associated with performing
small tasks in instrumentation snippets.

tool-specific -- path counter, xiaofeng's memory instructrion subset